\documentclass{article}
\title{CMSI 488 Homework 3}
\author{Anthony Khayat, David Kikuta and Allison Neyer}
\usepackage{url}
\begin{document}
\maketitle

\begin{enumerate}
    
\item [1.] Two test files that compile perfectly (no lexical, syntactic, or semantic errors). Together, these two files should contain all possible syntactic forms expressible in Roflkode. Use every operator, every keyword, every statement, etc. The programs can be nonsense, but they do have to be legal Roflkode. And remember, make sure you exercise everything.
\\
\\
See leetsyntax.rk and leetsyntax2.rk

\item [2.] Ten test files with syntax errors. Make them as "interesting" as you can. Each file should have one error.
\\
\\
See files beginning with ``synerror."

\item [3.] Thirteen test files that have no syntax errors but that DO have static semantic errors. Again, try to make them interesting. Look through the Roflkode definition and look for subtle rules.
\\
\\
See files beginning with ``semerror."

\end{enumerate}
    
\end{document}